\documentclass{emulateapj}
%\documentclass{aastex}
\submitted{{\it Submitted for publication in ApJL}}
\usepackage {graphicx}

\usepackage{amsmath} 
\usepackage{amssymb} 
\usepackage{graphics}
\usepackage{epsfig}  
\usepackage{float}
\bibliographystyle{apj}
\def\be{\begin{equation}}
\def\ee{\end{equation}}
\def\ba{\begin{eqnarray}}
\def\ea{\end{eqnarray}}

\newcommand{\documentname}{research note}
\newcommand{\avg}[1]{\langle{#1}\rangle}  
\newcommand{\skel}{$\beta$-Skeleton}  
\newcommand{\hMpc}{{\ifmmode{h^{-1}{\rm Mpc}}\else{$h^{-1}$Mpc }\fi}}  
\newcommand{\hGpc}{{\ifmmode{h^{-1}{\rm Gpc}}\else{$h^{-1}$Gpc }\fi}}  
\newcommand{\hmpc}{{\ifmmode{h^{-1}{\rm Mpc}}\else{$h^{-1}$Mpc }\fi}}  
\newcommand{\hkpc}{{\ifmmode{h^{-1}{\rm kpc}}\else{$h^{-1}$kpc }\fi}}  
\newcommand{\hMsun}{{\ifmmode{h^{-1}{\rm {M_{\odot}}}}\else{$h^{-1}{\rm{M_{\odot}}}$}\fi}}  
\newcommand{\hmsun}{{\ifmmode{h^{-1}{\rm {M_{\odot}}}}\else{$h^{-1}{\rm{M_{\odot}}}$}\fi}}  
\newcommand{\Msun}{{\ifmmode{{\rm {M_{\odot}}}}\else{${\rm{M_{\odot}}}$}\fi}}  
\newcommand{\msun}{{\ifmmode{{\rm {M_{\odot}}}}\else{${\rm{M_{\odot}}}$}\fi}}  
\newcommand{\kms}{{\ifmmode{{\mathrm{\,km\ s}^{-1}}}\else{\,km~s$^{-1}$}\fi}}
\newcommand{\bullb}{MACS J0025.4-1222}
\newcommand{\bulla}{1E0657---56} 
\newcommand{\bullg}{SL2S J08544-0121}
\shorttitle{Bullet Groups}
\shortauthors{Fern\'andez-Trincado et al.}

\begin{document} 

\title{The $\beta$-skeleton as a cosmological tool}
\author{J. E. Forero-Romero$^1$,
  Xiao-Dong Li$^2$  and Changbom Park$^{1}$}
\affil{$^1$ Departamento de F\'{i}sica, Universidad de los Andes,
  Cra. 1 No. 18A-10, Edificio Ip, Bogot\'a, Colombia\\ 
       $^2$ School of Physics, Korea Institute for Advanced Study,
  Heogiro 85, Seoul 130-722, Korea}
\email{je.forero@uniandes.edu.co}
\begin{abstract}
We explore the application of the $\beta$-skeleton, a concept from
geometric graph theory, as tool to describe the large scale structure
of the Universe.
\end{abstract}

\keywords{methods: numerical}

\section{Introduction}


\section{$\beta$-skeleton definition}

The \skel\ is a non-directed graph that can be defined on a set of
points in Euclidian space \citep{beta-def}.

We use the Neighboring Graph Library (NGL) \citep{Correa2011}to
compute the \skel\ of a set of points in 3D Euclidian space.  The NGL
is publicly available \footnote{{\texttt http://www.ngraph.org/}}.

\section{Simulation and Mock Catalogs}
\label{sec:simulation}

The data set is based on the Horizon Run 3 simulation data
\citep{HR3}. We use 27 mocks catalogs constructed to be close to the
number density of Luminous Red Galaxies in  


\section{Numerical Experiments}
\label{sec:experiments}

We construct the \skel\ from mock catalogs in redshift space that were
reconstructed in real space using cosmological parameters
$\Omega_m^{\prime}$, $\Omega_\Lambda^\prime$ and $\omega^{\prime}$,
where the prime indicates that their values are, in general, not the
values used in the simulation. In this \documentname\ we perform a
first study where these values correspond to the simulation values.

After each mock is reconstructed into real space we filter the number
of points using two parameters. The minimal mass of halos to be
included, $M_{\rm min}$; and the fraction of halos above this mass
that will be left in the final sample, $f_{\rm in}$. We use the
following set of pairs of values for $(M_{\rm min}, f_{\rm in})=\{(,),(,),(,),(,),(,),(,)\}$, this selection fixes the total number of particles in the catalog
to $\{,\}$. 

 





\section{Results}
\label{sec:results}

\section{Discussion}
\label{sec:discussion}

\section{Conclusions}
\label{sec:conclusions}

%\begin{figure*}
%\begin{center}
%\includegraphics[width=0.9\textwidth]{figure_1.eps}
%\end{center}
%\caption{}
%\end{figure*}


J.E.F-R acknowledges support from Vicerrector\'ia de
Investigaciones through a FAPA starting grant.

\bibliographystyle{apj}
\bibliography{references} 

\end{document}
